\documentclass[a4paper,8pt]{amsart}

\usepackage[osf,sc]{mathpazo}
\usepackage[a4paper]{geometry}
\usepackage{fullpage}
\usepackage{multicol}
\usepackage{todonotes}

\newcommand{\CMS}{\textsc{cms}}

\newcommand{\DB}{\textsc{db}}
\newcommand{\LS}{\textsc{ls}}
\newcommand{\FS}{\textsc{fs}}
\newcommand{\ES}{\textsc{es}}
\newcommand{\RS}{\textsc{rs}}
\newcommand{\WS}{\textsc{w}}
\renewcommand{\SS}{\textsc{ss}}
\newcommand{\CWS}{\textsc{cws}}
\newcommand{\AWS}{\textsc{aws}}
\newcommand{\RWS}{\textsc{rws}}

\newenvironment{squishlist}{%
  \begin{list}{\textbullet}%
    { \setlength{\itemsep}{0pt}%
      \setlength{\parsep}{3pt}%
      \setlength{\topsep}{3pt}%
      \setlength{\partopsep}{0pt}%
      \setlength{\leftmargin}{1.5em}%
      \setlength{\labelwidth}{1em}%
      \setlength{\labelsep}{0.5em} }%
}{\end{list}}

\newcommand{\id}[1]{\texttt{#1}}
\newcommand{\file}[1]{\texttt{#1}}

\title{A contest management system for programming contests}
\author{Matteo Boscariol, Stefano Maggiolo, Giovanni Mascellani}
\date{\today}


\begin{document}

\maketitle
\tableofcontents

\begin{multicols}{2}

  \section{Introduction}

  When organizing a programming contest, there are three main stages:
  \begin{squishlist}
  \item the first is to develop all the data that the assigned tasks
    need (i.e., statements, solutions, testcases, information on how
    to grade submissions, etc.);
  \item the second, that happens when the contest is onsite, is to
    properly configure the machines that the contestants are going to
    use during the contest, in particular with respect to network
    security;
  \item the third is to manage the actual contest (accepting and
    grading submissions, give feedback on them, display a live
    ranking, etc.).
  \end{squishlist}

  The aim of the \CMS\ project is to give a good answer to the third
  problem. Our goal is to develop a contest management system that is
  secure, extendable, adaptable to different situations, and easy to
  use.

  \section{General structure of the system}

  The system is organized in a modular way, with different services
  running (potentially) on different machines, and providing
  extendability via service replications on several machines.

  The state of the contest is wholly kept on a PostgreSQL database
  (\DB). At the moment, there is no way to use other SQL databases,
  because the Large Object (LO) feature of PostgreSQL is used. It is
  unlikely that in the future we will target different databases.

  As long as the \DB\ is operating correctly, all other services can
  be started and stopped independently without problems. This means
  that if a machine goes down, then the administrator can quickly
  replace it with an identical one, which will take its roles (without
  having to move information from the broken machine). Of course, this
  also means that if the \DB\ goes down, the system is unable to
  work. In critical contexts, it is the necessary to configure \DB\
  redundantly and being prepared to rapidly doing a fail-over in case
  something bad happens. The choice of PostgreSQL as the database to
  use should ease this part, since there are many different, mature
  and well-known solutions to provide such redundance a fail-over
  procedures.

  \subsection{Services of the system}

  The system is composed of the following functional parts, listed
  here with their main capabilities:
  \begin{squishlist}
  \item Log Service (\LS{}): receives all logs from the different
    services and offers them for remote inspection (logs are also kept
    on the machine where the service is running);
  \item Worker (\WS{}): can run compilation, evaluation and grading
    processes on the submissions in a safe and fair environment;
    arguably, this is the most delicate service, since it deals with
    untrusted code (submitted by users) and must do precise low-level
    operations (such as estimating the memory and CPU time consumed by
    an external program);
  \item Evaluation Service (\ES{}): builds a queue of jobs to do on the
    submissions and distribute them to the \WS{}'s;
  \item Scoring Service (\SS{}): transforms the output of the
    evaluation to an actual score of the submissions, and manages the
    external rankings;
  \item Contest Web Server (\CWS{}): interacts with the contestants,
    allowing to submit solutions and tests, to inspect previous
    submissions, to look at information about the contest, to ask
    questions, etc.;
  \item Admin Web Server (\AWS{}): exposes to the contest admins all
    the information about the contest, and gives the possibility to
    change its data;
  \item Resource Service (\RS): monitors and manages all other
    services running on the same machine (see subsection
    \ref{ssec:sharding}).
  \item Ranking Web Server (\RWS{}): serve publicly the real time
    ranking; \RWS\ is rather different from other services, because it
    is allowed (and encouraged) to run on a different network: you can
    run it on a remote server and the contest data will be streamed to
    it (there are multiple reasons for this decision: one is that the
    contest venue could have reduced bandwidth or lack a public IP to
    publish the results to the Web; the second is that, from a
    security viewpoint, it is not advisable to allow public access to
    the contest network).
  \end{squishlist}

  One is not forced to run everything: the only mandatory component is
  the \DB, while \LS\ is always optional (but strongly suggested);
  here are some examples.
  \begin{squishlist}
  \item Bare: only a \CWS{}; the system would accept submissions, but
    is unable to give feedback to the contestants and to the world,
    lacking something that evaluates the submissions.
  \item Minimal: add an \ES{} and a \WS{}; now we can evaluate
    submissions, so the contestants are informed of the state of the
    submissions, but no ranking is built.
  \item Private: add a \SS{} and a \AWS{}; now a ranking is built, and
    admins can interact with the system and make sure everything is
    allright.
  \item Public: add a \RWS{}, in order to show the live ranking.
  \item Redundant: add several other \WS{}'s and some other \CWS{}'s
    to manage big contest without having a long evaluation queue and
    high latency to access the contest web site.
  \end{squishlist}

  In case there are more \CWS\ services, the administrators should set
  up a load balancing mechanism. The repository contains an example
  configuration to build a simple one using nginx.

  As already said, these services can run on one or several machines
  (as long as they're connected together). The distribution of the
  services on the available machines usually depends on the number and
  characteristics of the available machines: an important point,
  though, is to keep each \WS\ on a dedicated machine, since the
  presence of other running programs can impact on the measured
  performance of the evaluated solutions. For the same motivation, it
  is advised to check that \WS\ machines do not have, for example,
  heavy cron jobs that may decrease the precision of the
  measurements. Remember also that most IOI-style contests require the
  evaluation to be run on computers identical to those used by the
  contestants.

  \subsubsection{Security considerations}

  With the exception of \RWS, there are no cryptographic or
  authentication schemes between the various services or between the
  services and the \DB. Thus, it is mandatory to keep the services on
  a dedicated network, properly isolating it via firewalls from
  contestants or other people's computers. This sort of operations,
  like also preventing contestants from communicating and cheating, is
  responsibility of the administrator and is not managed by
  \CMS\ itself.

  \subsection{Sharding mechanism}
  \label{ssec:sharding}

  Since the management of many different services on possibly many
  computers can be confusing for an administrator, there is a
  \emph{sharding} mechanism that is used to distinguish between
  different instances of the same system. The \CMS\ configuration file
  must be the same for all the services composing \CMS: the
  configuration file specifies one or more \emph{shards}, or
  instances, for every service. The shards are uniquely defined by
  their listening IP and port number. Moreover, there is one special
  \CMS\ service called ResourceService (\RS), which takes care of
  starting all other services running on the same machine and
  restarting them in case they crash. It also collects information
  about the resource usage on that machine
, reporting them for
  monitoring on the \AWS.

  \subsection{Interaction with the task development}

  The \AWS\ allows the modification of contest's and tasks'
  parameters. Nonetheless, while developing the tasks, it is much
  easier to work on the filesystem. We believe that every contest
  admin has his own preferred way of developing the tasks, hence we do
  not force any specific solution. Instead, we encourage the admins to
  use an \emph{importer\/}, that is a program that reads the data of a
  contest and of its tasks from the filesystem and create a
  corresponding contest.

  With the same goal in mind, the contest admin will run an
  \emph{exporter\/} at the end of the contest that will select the
  interesting data and write them in some format (e.g., suitable for
  the inclusion in a revision control system).

  We have currently two couples of importers/exporters (in the
  directory \file{cmscontrib}). The first one simply export and import
  everything (so the composition of the two is the identity) and are
  called \file{ContestImporter} and \file{ContestExporter}. These are
  useful for having a quick way to backup a contest and its state at
  some time.

  The second couple works specifically with the structure used by the
  Italian IOI team and by no means the exporter is useful to do
  backups, because it exports only the results and the submissions.
  Converting these program to use other structures should be tedious
  but easy.

  A slightly more complicated issue is that different contest admins
  can use different type of tasks.

  \subsubsection{An example}

  Before IOI 2010 in Canada, the tasks assigned at the IOI were mainly
  ``batch'' tasks, where the contestants have to submit a stand-alone
  source file; there had been sporadic cases of ``output-only'' tasks,
  where the contestants have the input cases and need to submit only
  the outputs to these inputs, and of ``interactive'' tasks, where the
  contestants submit a source file that will be linked to a given
  library.

  From IOI 2010, the distinction between batch and interactive tasks
  vanished, being them united in the ``programming'' tasks. In all
  these tasks, the contestants have to submit one or more source files
  that implement one or more functions; these source files will be
  linked to a given source file that implements the I/O and calls the
  contestants' functions.

  Note that the programming tasks may need to be treated differently
  in the evaluation. For example, when the statement of the task says
  that the contestants have to implement two functions that cannot
  communicate between each other, the evaluation needs to ensure this.

  These difference can be implemented \emph{inside\/} the system, as
  plugins. We believe that our internal structure of a task is
  flexible enough to manage most task types.


  \subsection{Network and computers set up}

  Details of setting up contestants' computers are not discussed here:
  briefly, the contest administrators will probably have to install a
  reasonably modern Linux distribution on some reference computer and
  then copy it on all other machines (after having installed the
  packages that the contestants may use). Probably the administrators
  will want a SSH server on any machine. Great care must be put in
  configuring security aspects, in order to prevent users from
  escalating privileges locally (i.e, becoming the root user) and
  using the network improperly (i.e., anything that is not legal
  communication with the \CWS).

  It is advised to install all contestants' machines and \WS's in 32
  bits mode, because the system is especially tested in that
  configuration. Anyway, it is a requirement to install each
  \WS\ consistently, i.e., either all in 32 bits mode or all in 64
  bits mode.

  The network should be set up partitioning it in two distinct
  subnets, one for the contestants and one for the \CMS\ machines. No
  communication between these two branches must be allowed, except
  HTTP requests directed towards the \CWS\ (or, even better, to an
  HTTP proxy for the \CWS). The \CMS\ network can also accommodate
  administrators' laptops. Of course it is expected that the firewall
  allows them to connect via SSH to contestants' computers.


  \section{Terms}

  Here we describe some terms that have a specific meaning inside
  \CMS{}.

  \begin{squishlist}
  \item[Importer (program).] An importer is a program that fills the
    configuration of the contest and \FS{} with the data needed for
    the contest. Contest admins may decide to do this step by hand.
  \item[Exporter (program).] An exporter is a program that exports
    some of the data of a contest from \CMS{} to the outside world,
    usually to save some information on a revision control system.
  \item[Compilation (process).] Compilation is done by \WS{}, and it
    consists of the compilation of the submission and the execution
    against a small number of testcases to ensure that it is
    syntactically correct. The result of the compilation is always
    visible to the contestant; if the submission fail the compilation,
    a message should be reported to hint at what part needs to be
    corrected. The procedure used to compile and run the submission is
    specified by the task type, and usually makes use of one or more
    managers.
  \item[Evaluation (process).] Evaluation is done by \WS{}, and it
    consists in the compilation of the submission and the execution
    against a large number of testcases; for each testcase, \WS{}
    produces an evaluation.
  \item[Manager (source code)] Managers are source code provided by
    the contest managers, that is going to be linked to the source
    code provided by the contestants, or used to evaluate the output
    of the submission after its execution against a testcase.
  \item[Outcome (data).] \WS{} produces a numerical value for each
    submission and each testcase, called outcome; this is not the
    correct value to use to build the ranking of the contest. The
    outcome is given by a manager of the task.
  \item[Score (data).] The score is a numerical value assigned by
    \ES{} to a submission, that depends on the outcomes of all
    testcases and all submissions sent before. The method used to
    compute the score is determined by the score type. This is the
    value to use to build the rank of the contest.
  \item[Score type (class).] A score type is a class that specifies
    how to translate the outcomes of the submissions into scores. Its
    behaviour is influenced by parameters specified in the
    configuration of the task.
  \item[Task type (class).] A task type is a class that specifies how
    to treat the files submitted by the contestant and the managers
    provided by the contest admins; usually, this turns out to be
    instructions for the compilation and linking of source code into
    one or more executable, and the instructions for the execution of
    these programs. A second duty of the task type is to decide how to
    merge a submission with the last one (sometimes it is useful to
    allow partial submissions, e.g. in output-only tasks).
  \item[Token.] A contestant can play a token against one of his
    submissions, in order to have an overview of the goodness of the
    submission. The actual result that the contestant can see is
    specified by the configuration. At the beginning of the contest,
    each contestant has a certain number of tokens; between two usage
    of a token, some time has to pass; moreover token regenerates
    after some time. These configuration may be tasks-oriented or
    contest-oriented
  \item[View (data).] A collection of derived data that is useful to
    more than one service and maybe heavy to compute. It is saved in
    \DB{} and available for all other services. For example, the
    outcome, and the rankings.
  \end{squishlist}
\end{multicols}
\end{document}
